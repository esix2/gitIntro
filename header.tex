\mode<presentation> {

% The Beamer class comes with a number of default slide themes
% which change the colors and layouts of slides. Below this is a list
% of all the themes, uncomment each in turn to see what they look like.

\usetheme{default}
%\usetheme{AnnArbor}
%\usetheme{Antibes}
%\usetheme{Bergen}
%\usetheme{Berkeley}
%\usetheme{Berlin}
%\usetheme{Boadilla}
%\usetheme{CambridgeUS}
%\usetheme{Copenhagen}
%\usetheme{Darmstadt}
%\usetheme{Dresden}
%\usetheme{Frankfurt}
%\usetheme{Goettingen}
%\usetheme{Hannover}
%\usetheme{Ilmenau}
%\usetheme{JuanLesPins}
%\usetheme{Luebeck}
%\usetheme{Madrid}
%\usetheme{Malmoe}
%\usetheme{Marburg}
%\usetheme{Montpellier}
%\usetheme{PaloAlto}
%\usetheme{Pittsburgh}
%\usetheme{Rochester}
%\usetheme{Singapore}
%\usetheme{Szeged}
%\usetheme{Warsaw}

% As well as themes, the Beamer class has a number of color themes
% for any slide theme. Uncomment each of these in turn to see how it
% changes the colors of your current slide theme.

%\usecolortheme{albatross}
%\usecolortheme{beaver}
%\usecolortheme{beetle}
%\usecolortheme{crane}
%\usecolortheme{dolphin}
\usecolortheme{dove}
%\usecolortheme{fly}
%\usecolortheme{lily}
% \usecolortheme{orchid}
%\usecolortheme{magenta}
%\usecolortheme{seagull}
%\usecolortheme{seahorse}
%\usecolortheme{whale}
%\usecolortheme{wolverine}

%\setbeamertemplate{footline} % To remove the footer line in all slides uncomment this line
%\setbeamertemplate{footline}[page number] % To replace the footer line in all slides with a simple slide count uncomment this line

%\setbeamertemplate{navigation symbols}{} % To remove the navigation symbols from the bottom of all slides uncomment this line
}

\usepackage[T1]{fontenc}  % access \textquotedbl
\usepackage{graphicx} % Allows including images
\usepackage{booktabs} % Allows the use of \toprule, \midrule and \bottomrule in tables
\usepackage{tikz}
\def\Check{\tikz\fill[color=green, scale=0.4] (0, 0.35) -- (0.25, 0) -- (1, 0.7) -- (0.25, 0.15) -- cycle;}
\def\shortside{0.15}
\def\longside{0.25}
\def\Cross{\tikz\fill[color=red,rotate=45,scale=0.4](0,0) -- (\longside,0) -- (\longside,\longside)
-- (\longside+\shortside,\longside) -- (\longside+\shortside,0) -- (2*\longside+\shortside,0)
-- (2*\longside+\shortside,-\shortside)-- (\longside+\shortside,-\shortside)
-- (\longside+\shortside,-\shortside-\longside) --  (\longside,-\shortside-\longside)
--(\longside,-\shortside) -- (0,-\shortside) -- cycle;}

\def\dsign{$\$$}
\newcommand\comm[1]{\\ \vspace{0.2cm}\hspace{0.5cm} \footnotesize{\color{blue}\dsign #1}}
\newcommand\subitem[1]{\\ \vspace{0.2cm}\hspace{0.5cm} #1}
\def\dhyphen{\texttt{-{}-}} % double hypen
%\def\qmark{\textquotedbl} %double qoutation mark
