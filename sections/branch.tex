 \begin{frame}{Branches in Git}{Creating, Displaying, and Switching}
  \begin{itemize}
\item Creating a branch
  \comm{git branch NEW\_BRANCH} (creates a new branch)
  \comm{git checkout -b BRANCH\_NAME} (creates and switch)
  \comm{git switch -c BRANCH\_NAME} (creates and switch, from Git 2.23)
\item Displaying branches
  \comm{git branch} (shows only local branches)
  \comm{git branch -r} (shows only remote branches)
  \comm{git branch -a} (shows all branches)
\item Switching between branches
  \comm{git checkout BRANCH\_NAME} (switches to another branch)
  \comm{git switch BRANCH\_NAME} (switches to another branch)
\end{itemize}
\end{frame}

 \begin{frame}{Branches in Git}{Comparing, Merging, Renaming, and Deleting}
  \begin{itemize}
\item Comparing two branches
  \comm{git branch BRANCH\_A..BRANCH\_B} (compares all files)
  \comm{git branch BRANCH\_A..BRANCH\_B SOME\_FILE} (compares only a given files)
\item Merging branches
  \comm{git merge BRANCH\_B} (merges branch b into branch a, you should be in branch a)
\item Renaming a branch
  \comm{git branch -m OLD\_NAME NEW\_NAME} 
\item Deleting a branch 
  \comm{git branch -d BRANCH\_FOR\_DELETION} (deletes a branch)
\end{itemize}
\end{frame}

\begin{frame}{Remote Branches}{Creating and Deleting}
  \begin{itemize}
\item Creating remotes branch in command line. It switches to the branch, as well.
\comm{git checkout -b origing/NEW\_BRANCH}
\item Deleting a remote branch 
  \comm{git branch -d --remotes origin/BRANCH\_FOR\_DELETION} 
\end{itemize}
\end{frame}
