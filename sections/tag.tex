\begin{frame}{Tag}{Create}
\begin{itemize}
  \item Tags are to refer to (important) moments of time
  \item They are used, e.g., as reference to release versions
\item There two types of tags
\begin{enumerate}
  \item \textbf{Lightweight} tag which is just a pointer to a specific commit.
    \comm{git tag -a TAG} (lightweight tag the current commit)
    \comm{git tag -a TAG HASH} (lightweight tag the given hash)
  \item \textbf{Annotated} tag which is checksummed, contains the tagger name, email, and date; have a tagging message; and can be signed and verified with GNU Privacy Guard (GPG).
    \comm{git tag -a TAG} (annotated tag the current commit)
    \comm{git tag -a TAG HASH} (annotated tag the given hash)
\end{enumerate}
\end{itemize}
\end{frame}

\begin{frame}{Tag}{List, Move, Delete, Push}
\begin{itemize}
\item Tags can be used as argument to other git commands,
  such as \emph{diff}, \emph{checkout}
    \comm{git tag} (shows all existing tags)
    \comm{git tag -l "v17*"} (shows all tags starting with v17)
    \comm{git tag -f TAG} (forces move existing tag to the current commit)
    \comm{git tag -d TAG} (deletes the tag)
  \item Git does not push the tag by default to the remote. You should tell
    git to do so, using
    \comm{git push \dhyphen tag}
\end{itemize}
\end{frame}
