\begin{frame}{Git Basics}{Cloning}
  \begin{itemize}
  \item Getting a remote repo:
    \comm{git init}
    \comm{git remote add origin REPO\_ADDRESS/REPO\_NAME.git}
    \comm{git pull origin master}
    \comm{git branch \dhyphen set-upstream-to=origin/master}
  \end{itemize}
  \begin{itemize}
  \item or the easy way $\rightarrow$ git clone
    \comm{git clone REPO\_ADDRESS/REPO\_NAME.git}
    \comm{git clone /home/ehszandi/Public/gitIntro.git} (This is an example)
  \end{itemize}
  Now the repository is cloned and you can work on it!
\end{frame}
    %\item \textbf{origin}: A shorthand name for the remote repo
    %  \comm{git remote show} (shows "origin" as output)
    %  \comm{git remote show origin} (shows detailed info on origin)
