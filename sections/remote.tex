\begin{frame}{Remote Repository}{Clone}
  \begin{itemize}
  \item Getting a remote repo:
    \comm{git init}
    \comm{git remote add origin REPO\_ADDRESS/REPO\_NAME.git}
    \comm{git pull origin master}
    \comm{git branch \dhyphen set-upstream-to=origin/master}
  \end{itemize}
  \begin{itemize}
  \item or the easy way $\rightarrow$ git clone
    \comm{git clone REPO\_ADDRESS/REPO\_NAME.git}
    \comm{git clone /home/ehszandi/Public/gitIntro.git} (This is an example)
  \end{itemize}
  Now the repository is cloned and you can work on it!
\end{frame}
\begin{frame}{Remote Repository}{Remote}
  \begin{itemize}
    \item \textbf{origin}: A shorthand name for the remote repo
      \comm{git remote show} (shows "origin" as output)
      \comm{git remote show origin} (shows detailed info on origin)
      \comm{git remote -v} (shows push/fetch address of the origin)
  \end{itemize}
\end{frame}
\begin{frame}{Remote Repository}{Fetch, Pull, Push}
  \begin{itemize}
      \comm{git fetch origin} (fetches the last changes in the origin/main)
\begin{itemize}
  \item Hint: Now your HEAD is one commit behind origin
\end{itemize}
\comm{git merge origin/main} (to merge origin/main into your HEAD)
\comm{git pulli origin} (fetches and merges at the same time)
\comm{git push origin/main) (pushed the last changes in your HEAD into origin/main)
  \end{itemize}
\end{frame}
