\begin{frame}{Hidden Area}{Stash}
  {\footnotesize Switching between branches or checking out older commits, while having unclean staging area, 
  are avoided or cause conflicts. Staged (but not commited) changes can hidden into the stash area to clean the  working area. Then, you can do whatever 
you want, while changes in the stash remians, until they are applied or deleted.}
\begin{itemize}
\item 
  \comm{git stash}  (stashes all staged changes)
  \comm{git stash save "Something"}  (stashes the changes with a label)
  \comm{git stash list} (shows the stash area, might be more than one entry)
  \comm{git stash apply stash@\{1\}} (applies the stash before most recent)
  \comm{git stash show stash@\{1\} -p} (shows content of $2^{\text{nd}}$ newest stash)
  \comm{git stash apply} (applies the most recent stash)
  \comm{git stash pop} (applies and then deletesi the stash)
  \comm{git stash clear} (clears the stash area)
  \comm{git stash drop stash@\{2\}} (deletes the $3^\text{rd}$ newest stash)
\end{itemize}
\end{frame}
