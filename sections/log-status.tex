\begin{frame}{Git Basics}{Status and Log}
\begin{itemize}
\item Status and log
\comm{git status} (Shows the status of the repo)
\comm{git log} (Shows the commit log on the current branch)
\comm{git log SOME\_BRNACH} (Shows the commit log on a specific branch)
\comm{git log \dhyphen all} (Shows the commit log on all branches)
\comm{git log -p} (Shows the commit log and the content difference of files per commit, combines 
git log and git diff)
\comm{git log \dhyphen decorate \dhyphen oneline \dhyphen graph \dhyphen all} (Very useful graph-like history)
\end{itemize}
\end{frame}
\begin{frame}{Git Basics}{Aliases}
\begin{itemize}
\item Git Aliases, some useful examples:
\comm{git config \dhyphen global alias.g \sq log \dhyphen decorate \dhyphen oneline \dhyphen graph \dhyphen all\sq} (makes "git g" an alias for the previous long command)
\comm{git config \dhyphen global alias.l log} (makes "git l" an alias for "git log")
\comm{git config \dhyphen global alias.loa \sq log \dhyphen oneline \dhyphen all\sq} (makes "git loa" an alias
for "git log \dhyphen oneline \dhyphen all")
\comm{git config \dhyphen global alias.s status} (makes "git s" an alias for "git status")
\comm{git config \dhyphen global alias.b status} (makes "git b" an alias for "git branch")
\comm{git config \dhyphen global alias.ch checkout} (makes "git ch" an alias for "git checkout")

\end{itemize}
\end{frame}
