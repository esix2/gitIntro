\documentclass{beamer}
\mode<presentation> {

% The Beamer class comes with a number of default slide themes
% which change the colors and layouts of slides. Below this is a list
% of all the themes, uncomment each in turn to see what they look like.

\usetheme{default}
%\usetheme{AnnArbor}
%\usetheme{Antibes}
%\usetheme{Bergen}
%\usetheme{Berkeley}
%\usetheme{Berlin}
%\usetheme{Boadilla}
%\usetheme{CambridgeUS}
%\usetheme{Copenhagen}
%\usetheme{Darmstadt}
%\usetheme{Dresden}
%\usetheme{Frankfurt}
%\usetheme{Goettingen}
%\usetheme{Hannover}
%\usetheme{Ilmenau}
%\usetheme{JuanLesPins}
%\usetheme{Luebeck}
%\usetheme{Madrid}
%\usetheme{Malmoe}
%\usetheme{Marburg}
%\usetheme{Montpellier}
%\usetheme{PaloAlto}
%\usetheme{Pittsburgh}
%\usetheme{Rochester}
%\usetheme{Singapore}
%\usetheme{Szeged}
%\usetheme{Warsaw}

% As well as themes, the Beamer class has a number of color themes
% for any slide theme. Uncomment each of these in turn to see how it
% changes the colors of your current slide theme.

%\usecolortheme{albatross}
%\usecolortheme{beaver}
%\usecolortheme{beetle}
%\usecolortheme{crane}
%\usecolortheme{dolphin}
\usecolortheme{dove}
%\usecolortheme{fly}
%\usecolortheme{lily}
% \usecolortheme{orchid}
%\usecolortheme{magenta}
%\usecolortheme{seagull}
%\usecolortheme{seahorse}
%\usecolortheme{whale}
%\usecolortheme{wolverine}

%\setbeamertemplate{footline} % To remove the footer line in all slides uncomment this line
%\setbeamertemplate{footline}[page number] % To replace the footer line in all slides with a simple slide count uncomment this line

%\setbeamertemplate{navigation symbols}{} % To remove the navigation symbols from the bottom of all slides uncomment this line
}

\usepackage[T1]{fontenc}  % access \textquotedbl
\usepackage{textcomp}     % access \textquotesingle
\usepackage{graphicx} % Allows including images
\usepackage{booktabs} % Allows the use of \toprule, \midrule and \bottomrule in tables
\usepackage{tikz}
\def\Check{\tikz\fill[color=green, scale=0.4] (0, 0.35) -- (0.25, 0) -- (1, 0.7) -- (0.25, 0.15) -- cycle;}
\def\shortside{0.15}
\def\longside{0.25}
\def\Cross{\tikz\fill[color=red,rotate=45,scale=0.4](0,0) -- (\longside,0) -- (\longside,\longside)
-- (\longside+\shortside,\longside) -- (\longside+\shortside,0) -- (2*\longside+\shortside,0)
-- (2*\longside+\shortside,-\shortside)-- (\longside+\shortside,-\shortside)
-- (\longside+\shortside,-\shortside-\longside) --  (\longside,-\shortside-\longside)
--(\longside,-\shortside) -- (0,-\shortside) -- cycle;}

\def\dsign{$\$$}
\newcommand\comm[1]{\\ \vspace{0.2cm}\hspace{0.5cm} \footnotesize{\color{blue}\dsign #1}}
\newcommand\subitem[1]{\\ \vspace{0.2cm}\hspace{0.5cm} #1}
\def\dhyphen{\texttt{-{}-}} % double hypen
\def\sq{\textquotesingle}   %single qoutation mark

%	TITLE PAGE

\title[Short title]{An Introduction to Git Talk} % The short title appears at the bottom of every slide, the full title is only on the title page

\author{Ehsan Zandi} % Your name
\institute[DTIT] % Your institution as it will appear on the bottom of every slide, may be shorthand to save space
{
Deutsche Telekom IT \\ % Your institution for the title page
\medskip
ehsan.zandi@telekom.de
}
\date{23.08.2021} % Date, can be changed to a custom date
%\date{\today} % Date, can be changed to a custom date

\begin{document}

\begin{frame}
\titlepage % Print the title page as the first slide
\end{frame}

\begin{frame}
\frametitle{Overview} % Table of contents slide, comment this block out to remove it
\tableofcontents % Throughout your presentation, if you choose to use \section{} and \subsection{} commands, these will automatically be printed on this slide as an overview of your presentation
\end{frame}

%	PRESENTATION SLIDES

\section{Git vs SVN}

% \subsection{Subsection Example} % A subsection can be created just before a set of slides with a common theme to further break down your presentation into chunks

\begin{frame}{Git vs SVN}
\begin{itemize}
  \item Git is a fully distributed version control system (VCS)
  \item Each user (PC/Laptop) is an exact clone of the remote repository
    \begin{itemize}
      \item Each user is a repository (log, revert, merge, branch, etc)
      \item No network connection required, except to sync with central repo (pull/push/fetch)
      \item merge and rebasing can be done offline
    \end{itemize}
  \item Git is much faster than SVN
  \item Git's repositories are much smaller than SVN
  \item Git's branches are much simpler and less resource heavy than SVN
  \item Git is much better in branch auditing and merge handling
  \item As many backups as the number of users ()
  \item Content integrity using SHA-1 hash
  \end{itemize}
\end{frame}

\begin{frame}{Git vs SVN}
  \begin{figure}
    \begin{center}
    \includegraphics[width=0.5\linewidth]{pics/git-vs-svn.png}
    \caption{\small Centralized vs distributed VCS (Source: www.git-tower.com)}
  \end{center}
\end{figure}
\end{frame}

\begin{frame}{Git vs SVN}
  \begin{center}
  \begin{tabular}{ c || c | c  }
         & SVN & Git \\ \hline\hline
    \textbf{License} & Open-source (Apache) & GNU \\
    \textbf{Distributed-ness} & Centralized & Fully Distributed  \\
    \textbf{Speed} & \Cross & \Check \\
    \textbf{Storage} & \Cross & \Check \\
    \textbf{Integrity Guarantee} & \Cross & \Check \\
    \textbf{Brnaching \& merging} & \Cross & \Check \\
    \textbf{Stashing} & \Cross & \Check \\

      \end{tabular}
    \end{center}
\end{frame}
\section{Git Basics}

\begin{frame}{Git Basics}{Architecture}
  \begin{itemize}
    \item Remote: The central repo (on a host machine/server, e.g., Github or Gitlab)
      {\color{red}$\rightarrow$ is identified by the alias "origin"}
    \item Repository: The local repo (.git sub-directory inside your working directory), created by "git init" or "git clone", i.e., ceartion/clonining
    \item Index or staging area: State between the working directory and repository (after modifying and before commiting)
    \item Workspace or working directory: your local machine, including all directories, sub-directories, and files of your project
  \end{itemize}
  \begin{figure}
    \begin{center}
    \includegraphics[width=0.8\linewidth]{pics/architecture.png}
    \vspace{-0.3cm}
    \caption{\small Git architecture (Source: www.stackoverflow.com)}
  \end{center}
\end{figure}

\end{frame}

\begin{frame}{Git Basics}{Definitions}
  \begin{itemize}
    \item \textbf{origin}: A shorthand name for the remote repo
      \comm{git remote show} (shows "origin" as output)
      \comm{git remote show origin} (shows detailed info on origin)
\item \textbf{branch}: A movable pointer to a commit
\item \textbf{master (or sometimes main)}: Default name of the (first) branch: can be changed
\item \textbf{HEAD}: A special pointer that tells on (the tip of) which branch you are.
\item \textbf{origin/HEAD}: A special pointer that tells on which branch the remote repo is.
\end{itemize}
\end{frame}

\begin{frame}{Git Basics}{Initializing a repo}
  \begin{itemize}
    \item Creating a local repo (without any remote)
    \comm{git init} (creates .git sub-directory)
    \comm{echo "hello world." $>>$ firstFile.txt} (makes changes in  working area)
    \comm{git status} (You see that your commit has some hash value)
    \comm{git add firstFile.txt} (puts your changes into staging area)
    \comm{git status} (You see that your commit has some hash value)
    \comm{git commit -m "A proper message"} (Now you have your first commit)
    \comm{git status} (A clean repo and one commit with a hash value)
  \end{itemize}
\end{frame}

\begin{frame}{Git Basics}{Add/Commit}
  \begin{itemize}
    \item \textbf{git add}: To add a new file or modified into the staging (index) area. It makes 
      the changes ready for commiting.
      \comm{git add FILE\_NAME}
      \comm{git add .} (adds all the changes current directory and sub-directories)
    \item \textbf{git commit}: To put the staged files into the (local) repo. Such changes can be tracked, i.e.,
      revert, log, etc.
      \comm{git commit -m "A proper message"}
  \end{itemize}
  \begin{figure}
    \begin{center}
    \includegraphics[width=0.55\linewidth]{pics/areas.png}
    \caption{\small Git areas (Source: https://git-scm.com)}
  \end{center}
\end{figure}
\end{frame}

%\begin{frame}{Remote Repository}{Clone}
  \begin{itemize}
  \item Getting a remote repo:
    \comm{git init}
    \comm{git remote add origin REPO\_ADDRESS/REPO\_NAME.git}
    \comm{git pull origin master}
    \comm{git branch \dhyphen set-upstream-to=origin/master}
  \end{itemize}
  \begin{itemize}
  \item or the easy way $\rightarrow$ git clone
    \comm{git clone REPO\_ADDRESS/REPO\_NAME.git}
    \comm{git clone /home/ehszandi/Public/gitIntro.git} (This is an example)
  \end{itemize}
  Now the repository is cloned and you can work on it!
\end{frame}
\begin{frame}{Remote Repository}{Remote}
  \begin{itemize}
    \item \textbf{origin}: A shorthand name for the remote repo
      \comm{git remote show} (shows "origin" as output)
      \comm{git remote show origin} (shows detailed info on origin)
      \comm{git remote -v} (shows push/fetch address of the origin)
  \end{itemize}
\end{frame}
\begin{frame}{Remote Repository}{Fetch, Pull, Push}
  \begin{itemize}
      \comm{git fetch origin} (fetches the last changes in the origin/main)
\begin{itemize}
  \item Hint: Now your HEAD is one commit behind origin
\end{itemize}
\comm{git merge origin/main} (to merge origin/main into your HEAD)
\comm{git pulli origin} (fetches and merges at the same time)
\comm{git push origin/main) (pushed the last changes in your HEAD into origin/main)
  \end{itemize}
\end{frame}


% \begin{frame}{Git Basics}{Definitions}
%  \begin{itemize}
%    \item \textbf{origin}: A shorthand name for the remote repo
%      \comm{git remote show} (shows "origin" as output)
%      \comm{git remote show origin} (shows detailed info on origin)
%\item \textbf{branch}: A movable pointer to a commit
%  \comm{git branch NEW\_BRANCH} (create a new branch)
%  \comm{git branch -d BRANCH\_FOR\_DELETION} (delete a branch)
%  \comm{git branch -a} (display all branches)
%  \comm{git checkout BRANCH\_NAME} (switch to another branch)
%\item \textbf{master (or sometimes main)}: Default name of the (first) branch: can be changed
%  \comm{git branch -r} (shows only remote branches)
%  \comm{git branch -a} (shows all branches)
%\item \textbf{HEAD}: A special pointer that tells on which branch you are.
%  \comm{git log \dhyphen decorate \dhyphen oneline \dhyphen graph \dhyphen all}
%\item \textbf{HEAD}: A special pointer that tells on which branch you are.
%\end{itemize}
%\end{frame}
\end{document}
